\documentclass[12pt]{article}

\usepackage{graphicx}
\usepackage{paralist}
\usepackage{listings}
\usepackage{booktabs}
\usepackage{amsfonts}

\oddsidemargin 0mm
\evensidemargin 0mm
\textwidth 160mm
\textheight 200mm

\pagestyle {plain}
\pagenumbering{arabic}

\newcounter{stepnum}

\title{Assignment 2 Report}
\author{Ninos Yomo | yomon}
\date{\today}

\begin {document}

\maketitle

\section{Testing of the Original Program}

My approach to testing was simple and systematic. I would test each individual module,
and from each module, I would test the functionality of its methods. For each method,
if applicable, I would test a general case, a boundary case and an exception case. The
testing of boundary cases was a rational decision as I was able to uncover that my
index() function from SeqServices.py was not fully correct as it would fail on the 
boundary case. The exception cases were required to confirm that the module was correctly
raising errors when expected. Other than my boundary case for index(), the functionality 
of my other methods were correct.

\section{Results of Testing Partner's Code}

The SeqServices passed all cases. Although my test module raised an error for index(),
my partner's implementation was correct.$\\$$\\$

The CurveADT passed in all cases except for eval() because of the different implementation
of the local function, where my test module assumes scipy was used and implements a scipy
method that would crash for different implementations.$\\$$\\$

The getC method in Data does not work for indexs of 0 since my partner defined it to be
out of bounds even though indexing starts at 0 and the specification defined it in bounds.
The eval method for Data did not work as the calculation did not output a value. Finally,
the slice method did not work for the same reason eval in CurveADT did not work, the 
different implementation of the local function f caused .take(0) to raise an error.

\section{Discussion of Test Results}

\subsection{Problems with Original Code}

With my program, my exceptions were done differently then what most people would do.
Instead of raising an exception while passing a string message, I would raise the exception
and have the exceptions class come with a preset message. This would prove to be 
incompatible with my partner's code as they would attempt to pass a string message.$\\$$\\$
My implementation for index() in SeqServices was incorrect for the right boundary case since
it returns an index of the last element when it should be the index of the element before it
to maintain the definition of the specification.

\subsection{Problems with Partner's Code}

At first, the test module would not work because of the way SeqServices.py was implemented
by my partner. A class was made where the methods were defined to be static, which treated
the SeqServices as an abstract object while I defined it to simply be a module of usable 
methods. The requirements specification did not explicitly say if it was to be an abstract
object or not. However from context, it is implied that it is not a abstract object. $\\$$\\$

Another problem was the inconsistency of the import statements. I used methods
from imports by calling from x import *, whereas my partner would import specifically what was 
needed, because of this, the way those methods are called are different and the test module
would crash. $\\$$\\$

My partner incorrectly implemented getC from Data by treating the 0 index as out of bounds
even though the requirements specification defined it to be in bounds.

\subsection{Problems with Assignment Specification}
The assignment specification gave too much freedom when it came to the implementation of the 
local function f in CurveADT. We could have implemented it using scipy or SeqServices. However,
for the scipy implementation, the function .take(0) was used to convert the returned array into
a single value, this would work for all scipy implementations however, those who chose to implement 
using SeqServices as did my partner would find that the test would fail. $\\$$\\$

The specification did not define a consistent way to import the modules used in other modules.
Because of this, we used different ways of importing module that effected the way we wrote our
code and thus raised errors during testing because of the inconsistency.$\\$$\\$

Aside from these slight differences, the formality of the requirements specification of A2 compared
to A1 greatly improved the overall ability to implement the program. With it, many ambiguous details
were covered which allowed for consistent implementations. An example is the use of discrete math
notation to describe the functionality of methods, this allows for no misinterpretation, unlike the
previous assignment who's ambiguous description caused inconsistency in implementations.
\section{Answers}

\begin{enumerate}

\item What is the mathematical specification of the \texttt{SeqServices} access
  program isInBounds(X, x) if the assumption that X is ascending is removed?

If the assumption that the sequence is ascending was removed, this would mean the
current mathematical specification of: $X_0 \leq x \leq X_{|X|-1}$ would be incorrect.
The new specification must explicitly state the minimum and maximum values of the 
sequence instead of relying on the positional application of the ascending property.
This would be: $min(X) \leq x \leq max(X)$ where min and max represent the minimum
and maximum values of the sequence X respectively. An alternative way using no 
functions would be to say: $\exists x:\mathbb{R} \bullet (\neg(\forall y:\mathbb{R}|y \in S \bullet x<y)\land \neg(\forall z:\mathbb{R}|z \in S \bullet x>z))$
for the input argument x and the sequence S. If this is true, that means that the argument
x is not less than all of the elements in the sequence which means it is greater than 
at least one element, this proves that it is within the lower bounds. The second half
states that it is not greater than all the elements meaning it is less than at least
one element and therefore proves it is within the upper bounds. Since it is within
both boundaries, it must be within the sequence.

\item How would you modify \texttt{CurveADT.py} to support cubic interpolation?

To implement cubic interpolation, this would require that CurveADT.py has its 
$MAX\_ORDER$ value become 3 instead of 2 to allow for cubic functions to be inputted. This
would be the only requirement since the rest will be handled by scipy.interp1d()
because of the fact that it supports cubic interpolations. 

\item What is your critique of the CurveADT module's interface.  In particular,
  comment on whether the exported access programs provide an interface that is
  consistent, essential, general, minimal and opaque.

 Firstly, the interface is certainly essential as no existing method's functionality
 can be recreated with another in the same module. It is also minimal since all of
 the methods in CurveADT strictly perform only their sole intended task. However, 
 the boundaries could have been better defined as some methods may not work for them
 like solving for the differentials on the boundary points since it requires you to
 evaluate for a point out of bounds.

\item What is your critique of the Data abstract object's interface.  In
  particular, comment on whether the exported access programs provide an
  interface that is consistent, essential, general, minimal and opaque.

 The most obvious critique of Data is the lack of a method that deletes a
 curve from Data. It is known that Data has a very limited capacity of 10 
 curves, because of this, it would be useful for the module to implement a 
 deletion method to make room for new curves. Another detail is the lack of 
 a getter method for the size of Data. Without it, checking for the Full() 
 exception is made awkward as it is done by manually examining the length
 of Data, while a simpler size() function can help clean up the code and check
 for the Full() exception ahead of time.

\end{enumerate}

\newpage

\lstset{language=Python, basicstyle=\tiny, breaklines=true, showspaces=false,
  showstringspaces=false, breakatwhitespace=true}

\def\thesection{\Alph{section}}

\section{Code for CurveADT.py}

\noindent \lstinputlisting{../src/CurveADT.py}

\newpage

\section{Code for Data.py}

\noindent \lstinputlisting{../src/Data.py}

\newpage

\section{Code for SeqServices.py}

\noindent \lstinputlisting{../src/SeqServices.py}

\newpage

\section{Code for Plot.py}

\noindent \lstinputlisting{../src/Plot.py}

\newpage

\section{Code for Load.py}

\noindent \lstinputlisting{../src/Load.py}

\newpage

\section{Code for Partner's CurveADT.py}

% Uncomment the line below when partner files have been pushed to your repo
%\noindent \lstinputlisting{../partner/CurveADT.py}

\newpage

\section{Code for Partner's Data.py}

% Uncomment the line below when partner files have been pushed to your repo
%\noindent \lstinputlisting{../partner/Data.py}

\newpage

\section{Code for Partner's SeqServices.py}

% Uncomment the line below when partner files have been pushed to your repo
%\noindent \lstinputlisting{../partner/SeqServices.py}

\newpage

\section{Makefile}

\lstset{language=make}
\noindent \lstinputlisting{../Makefile}

\end {document}
\grid
